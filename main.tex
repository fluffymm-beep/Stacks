\documentclass{article}
\usepackage{amsmath}
\usepackage{amsfonts}
\usepackage{hyperref}
\usepackage{listings}

\title{Stacks in Python: Concepts, Implementations, and Examples}
\author{}
\date{}

\begin{document}

\maketitle

\section{Introduction}

A stack is a linear data structure that follows the LIFO (Last-In-First-Out) principle. This means that the last element added to the stack is the first one to be removed. Stacks are used in various applications, such as managing function calls, implementing Undo/Redo functionality, evaluating mathematical expressions, and recursive algorithms.

In this article, we will explore the concepts of stacks, different implementation methods in Python, performance comparisons, and practical examples.

\section{Main Stack Operations}

A stack typically includes the following operations:

\begin{itemize}
    \item \textbf{Push}: Add an element to the top of the stack.
    \item \textbf{Pop}: Remove and return the top element of the stack.
    \item \textbf{Peek (Top)}: View the top element of the stack without removing it.
    \item \textbf{isEmpty}: Check if the stack is empty.
    \item \textbf{Size}: Return the number of elements in the stack.
\end{itemize}

\textbf{General Structure of a Stack:}

\begin{verbatim}
Stack:
|  30  | ← Top of the stack
|  20  |
|  10  | ← First input element
---------
\end{verbatim}

\section{Methods of Implementing a Stack in Python}

\subsection{Using Python Lists}

Python lists natively support stack operations. The \texttt{append()} method is used for Push and \texttt{pop()} for Pop.

\textbf{Example:}

\begin{lstlisting}[language=Python]
stack = []
stack.append(10)  # Push: [10]
stack.append(20)  # Push: [10, 20]
stack.append(30)  # Push: [10, 20, 30]

print(stack.pop())  # Pop: 30 → Stack: [10, 20]
print(stack[-1])    # Peek: 20
print(len(stack))   # Size: 2
\end{lstlisting}

\textbf{Advantages and Limitations:}

\begin{itemize}
    \item \textbf{Simplicity}: No need for additional libraries.
    \item \textbf{High performance}: \texttt{append()} and \texttt{pop()} operations at the end of the list are O(1).
    \item \textbf{Low performance in pop(0)}: If the first element of the list is removed, the operation becomes O(n).
\end{itemize}

\subsection{Using \texttt{collections.deque}}

The \texttt{collections} module provides a deque (double-ended queue) data structure, which is more efficient for implementing stacks.

\textbf{Example:}

\begin{lstlisting}[language=Python]
from collections import deque

stack = deque()
stack.append(10)  # Push
stack.append(20)  # Push
stack.append(30)  # Push

print(stack.pop())  # Pop: 30
print(stack[-1])    # Peek: 20
print(len(stack))   # Size: 2
\end{lstlisting}

\textbf{Advantages:}

\begin{itemize}
    \item \textbf{Better performance}: All operations are O(1).
    \item \textbf{More stable} than Python lists for processing large data.
\end{itemize}

\subsection{Using \texttt{queue.LifoQueue} (Thread-Safe Stack)}

The \texttt{queue} module provides the \texttt{LifoQueue} class, which is suitable for multi-threaded (thread-safe) programs.

\textbf{Example:}

\begin{lstlisting}[language=Python]
from queue import LifoQueue

stack = LifoQueue()
stack.put(10)
stack.put(20)
print(stack.get())  # Output: 20
\end{lstlisting}

\textbf{Advantages:}

\begin{itemize}
    \item \textbf{Thread safety}.
    \item \textbf{More complexity} compared to deque and lists.
\end{itemize}

\subsection{Implementing a Stack with a Custom Class}

To have more control over stack operations, a custom class can be implemented.

\textbf{Example:}

\begin{lstlisting}[language=Python]
class Stack:
    def __init__(self):
        self.items = []
    
    def push(self, item):
        self.items.append(item)
    
    def pop(self):
        if not self.is_empty():
            return self.items.pop()
        raise IndexError("Stack is empty!")
    
    def peek(self):
        if not self.is_empty():
            return self.items[-1]
        raise IndexError("Stack is empty!")
    
    def is_empty(self):
        return len(self.items) == 0
    
    def size(self):
        return len(self.items)

# Using the class
s = Stack()
s.push(100)
s.push(200)
print(s.pop())  # Output: 200
print(s.peek())  # Output: 100
\end{lstlisting}

\textbf{Advantages:}

\begin{itemize}
    \item \textbf{Extensibility}: Ability to add new methods like search.
    \item \textbf{Better encapsulation}.
    \item \textbf{More complexity} compared to previous methods.
\end{itemize}

\section{Real-World Applications of Stacks}

\subsection{Checking Parentheses Balance in Mathematical Expressions}

Stacks are used to check the balance of parentheses in mathematical expressions.

\textbf{Example:}

\begin{lstlisting}[language=Python]
def is_balanced(expression):
    stack = []
    pairs = {')': '(', ']': '[', '}': '{'}
    for char in expression:
        if char in '([{':
            stack.append(char)
        elif char in ')]}':
            if not stack or stack.pop() != pairs[char]:
                return False
    return not stack

print(is_balanced("{[()]}"))  # True
print(is_balanced("{[(])}"))  # False
\end{lstlisting}

\subsection{Implementing Undo/Redo Functionality}

Two stacks can be used to manage Undo and Redo actions.

\textbf{Example:}

\begin{lstlisting}[language=Python]
undo_stack = []
redo_stack = []

def do_action(action):
    undo_stack.append(action)

def undo():
    if undo_stack:
        action = undo_stack.pop()
        redo_stack.append(action)

def redo():
    if redo_stack:
        action = redo_stack.pop()
        undo_stack.append(action)
\end{lstlisting}

\section{Comparing Stack Implementation Methods in Python}

\section{Conclusion}

Stacks are fundamental data structures used in various tasks like function management, recursive algorithms, text processing, and controlling Undo/Redo operations. In Python, depending on the project requirements, you can implement stacks using lists, deque, LifoQueue, or a custom class.

The best method depends on your needs:

\begin{itemize}
    \item If you need simplicity, use lists.
    \item If you have large data, deque is more suitable.
    \item If you want thread-safety, choose LifoQueue.
    \item If you need high flexibility, build a custom class.
\end{itemize}

\end{document}

